%%%%%%%%%%%%%%%%%%%%%%%%%%%%%%%%%%%%%%%%%%%%%%%%%%%%%%%%%%%%%%%%%%%%%%%%%%%%%%%%
% AI3603 Billiards Project Report
%%%%%%%%%%%%%%%%%%%%%%%%%%%%%%%%%%%%%%%%%%%%%%%%%%%%%%%%%%%%%%%%%%%%%%%%%%%%%%%%

\documentclass[letterpaper, 10 pt, conference]{ieeeconf}

\IEEEoverridecommandlockouts
\overrideIEEEmargins

\usepackage[utf8]{inputenc}
\usepackage[T1]{fontenc}
\usepackage{graphicx}
\usepackage{amsmath}
\usepackage{amssymb}
\usepackage{booktabs}
\usepackage{hyperref}
\usepackage{algorithm}
\usepackage{algorithmic}
\usepackage{subcaption}

\title{\LARGE \bf
Learning to Play Billiards: \\
A Journey Through Reinforcement and Imitation Learning
}

\author{Abigail Jin\\
\textit{Department of Computer Science and Engineering}\\
\textit{Shanghai Jiao Tong University}\\
Shanghai, China
}

\begin{document}

\maketitle
\thispagestyle{empty}
\pagestyle{empty}

%%%%%%%%%%%%%%%%%%%%%%%%%%%%%%%%%%%%%%%%%%%%%%%%%%%%%%%%%%%%%%%%%%%%%%%%%%%%%%%%
\begin{abstract}

This report documents my systematic exploration of learning-based approaches for the billiards AI challenge. Starting from a physics-based baseline (60\% win rate), I developed a complete Soft Actor-Critic (SAC) reinforcement learning framework with curriculum learning, followed by an extensive imitation learning pipeline that collected 1.78 million expert demonstrations and experimented with five neural network architectures. While these learning approaches did not surpass the MCTS expert, the journey revealed fundamental insights about why search-based methods excel in high-precision physics tasks: the multi-modal action distribution problem in imitation learning, and the sparse reward challenge in reinforcement learning. The final MCTS solution by my teammate achieved 80.6\% against BasicAgentPro.

\end{abstract}

%%%%%%%%%%%%%%%%%%%%%%%%%%%%%%%%%%%%%%%%%%%%%%%%%%%%%%%%%%%%%%%%%%%%%%%%%%%%%%%%
\section{INTRODUCTION}

Billiards presents a fascinating and challenging domain for artificial intelligence research. The game requires precise control over a continuous action space with five parameters---cue velocity, horizontal and vertical shooting angles, and two spin components---while navigating complex physics dynamics involving ball collisions, cushion rebounds, and friction effects. Beyond mechanical precision, successful play demands tactical awareness (selecting optimal shots from multiple candidates) and strategic planning (positioning for future shots, avoiding game-ending fouls).

This combination of challenges makes billiards an interesting test case for modern machine learning methods. Can deep reinforcement learning discover effective strategies through self-play? Can neural networks learn to imitate expert behavior from demonstration data? Or do the unique characteristics of billiards---its high precision requirements, sparse reward structure, and complex sequential constraints---favor alternative approaches?

This project systematically investigates these questions through extensive experimentation with three approaches.

First, I developed a \textbf{physics-based baseline agent} using geometric shot calculation and physics simulation, achieving 60\% win rate against the provided BasicAgent. This baseline establishes the difficulty of the task and provides insight into the core challenges: precise angle calculation, obstacle avoidance, and robustness to execution noise. I also analyzed the \textbf{robustness of opponent agents} through experiments with a weak random agent, discovering that even competent baseline agents lose 18--25\% of games through ``self-defeat''---pocketing the 8-ball prematurely or scratching during critical shots.

Second, I developed a complete \textbf{Soft Actor-Critic (SAC) reinforcement learning} framework with curriculum learning, parallel training, and extensive reward engineering. Initial experiments against BasicAgent achieved 0\% win rate, leading to a pivot toward self-play with simplified ball configurations. While the agent learned to play a 4-ball game with reasonable competence when strict draw penalties were applied, scaling to 6 balls immediately caused regression to a reward-induced degenerate equilibrium. The fundamental challenge proved to be the sequential constraint that the 8-ball must be pocketed last---a temporal dependency that pure RL struggles to learn implicitly.

Third, I built an extensive \textbf{imitation learning pipeline} to address a practical constraint: the competition requires real-time decision-making with a 3-minute time limit per game (1.5 minutes per player). While my teammate's MCTS agent achieved excellent performance (96.9\% win rate), it exceeded the time budget. I collected 1.78 million state-action pairs from this MCTS expert and experimented with five neural network architectures of increasing sophistication, aiming to approximate the expert's decision quality while achieving real-time inference speed. Despite achieving low validation loss, the best model achieved only 35.5\% win rate against BasicAgent, compared to the expert's 96.9\%. The fundamental limitation is the multi-modal action problem: when multiple optimal actions exist for similar states, MSE-based training causes the network to predict averaged actions that are themselves suboptimal.

Each approach built upon lessons learned from the previous, ultimately revealing why Monte Carlo Tree Search (MCTS) remains the superior approach for this domain. The negative results from learning-based methods are themselves valuable: they empirically demonstrate the boundaries of current techniques and highlight the importance of matching algorithm choice to problem structure.

The remainder of this report is organized as follows. Section II describes the baseline agents and opponent analysis. Section III details the reinforcement learning experiments and failure analysis. Section IV covers the imitation learning pipeline and its limitations. Section V provides comparative analysis across all approaches. Section VI concludes with key insights and directions for future work.

%%%%%%%%%%%%%%%%%%%%%%%%%%%%%%%%%%%%%%%%%%%%%%%%%%%%%%%%%%%%%%%%%%%%%%%%%%%%%%%%
\section{BASELINE AGENTS AND OPPONENT ANALYSIS}

Before developing learning-based approaches, I established baseline performance and analyzed the opponents we compete against. This preliminary work provided crucial insights that shaped the subsequent experiments.

\subsection{Physics-Based Baseline Agent}

I implemented a physics-based agent to understand the fundamental requirements for billiards play. The agent follows a classical approach: for each target ball, it computes the ``ghost ball'' position---the location where the cue ball must be placed to pocket the target along a straight line to the pocket. The shooting angle is then calculated as the direction from the current cue ball position to this ghost ball position.

The agent evaluates all possible target-pocket combinations (up to 42 in the worst case: 7 target balls times 6 pockets), computing cut angles and checking for obstructions along both the cue-to-target and target-to-pocket paths. Shots with extreme cut angles (greater than 60 degrees) or blocked paths are rejected. The remaining candidates are ranked by cut angle magnitude, with straighter shots preferred.

For each candidate shot, the agent uses the \texttt{pooltool} physics engine to simulate the outcome, verifying that the target ball actually enters the pocket and that no fouls occur (scratching, wrong ball first contact, etc.). This simulation-based verification is crucial because geometric calculations alone cannot account for complex collision dynamics, especially when multiple balls are in proximity.

The agent achieved 60\% win rate against BasicAgent, establishing a baseline understanding of the problem's requirements. The 40\% loss rate highlighted several challenges: the greedy one-shot-ahead approach often leaves the cue ball in poor position for subsequent shots, difficult cuts are frequently missed due to execution noise, and defensive considerations are entirely absent.

This work provided the foundation for my teammate's MCTS agent, which extends the basic shot evaluation with look-ahead search to consider multi-shot sequences and position play.

\subsection{Opponent Robustness Analysis}

To better understand the baseline agents we compete against, I designed a controlled experiment probing their robustness to weak opponents. The hypothesis was that a truly robust agent should win nearly 100\% of games against an opponent that barely functions.

I implemented a \texttt{RandomAgent} that makes intentionally weak shots: velocity uniformly sampled from 0.5--1.5 m/s (too slow for reliable pocketing), horizontal angle uniformly random from 0--360 degrees, vertical angle random from 0--90 degrees, and random spin parameters. This agent almost never pockets balls intentionally and provides minimal challenge.

\begin{table}[h]
\caption{Robustness Test: Baseline Agents vs RandomAgent (1000 games)}
\label{tab:robustness}
\begin{center}
\begin{tabular}{|l|c|c|c|}
\hline
\textbf{Agent} & \textbf{Win Rate} & \textbf{Loss Rate} & \textbf{Avg Shots} \\
\hline
BasicAgent & 74.6\% & 25.1\% & 42.6 \\
BasicAgentPro & 81.7\% & 18.2\% & 23.9 \\
\hline
\end{tabular}
\end{center}
\end{table}

The results in Table~\ref{tab:robustness} were surprising: even against this nearly-passive opponent, BasicAgent loses 25\% and BasicAgentPro loses 18\% of games. Since \texttt{RandomAgent} almost never pockets balls intentionally, these losses must occur when the baseline agent defeats itself.

Analysis of individual game logs revealed three primary failure modes. First, \textbf{premature 8-ball pocketing}: the agent pockets the 8-ball before clearing all own balls, resulting in immediate loss. This occurs when the 8-ball happens to be near a pocket and the agent's shot inadvertently sends it in. Second, \textbf{scratching on the 8-ball shot}: even when the agent correctly saves the 8-ball for last, it sometimes scratches (pockets the cue ball) on the winning shot, which results in loss by rule. Third, \textbf{other foul-related losses}: various sequences of fouls can lead to loss even against a non-pocketing opponent.

This 18--25\% ``self-defeat'' rate reveals a fundamental limitation of greedy optimization approaches: they pursue high-reward pocketing actions without adequately considering game-ending foul risks. A shot that pockets the target ball with 95\% probability but risks the 8-ball with 5\% probability has negative expected value if the 8-ball risk means immediate loss.

This insight directly motivated our exploration of look-ahead methods like MCTS, which can simulate action consequences and reject risky shots before execution. It also suggested that learning-based approaches might benefit from explicitly modeling foul risks, though as we shall see, this proved difficult to achieve in practice.

%%%%%%%%%%%%%%%%%%%%%%%%%%%%%%%%%%%%%%%%%%%%%%%%%%%%%%%%%%%%%%%%%%%%%%%%%%%%%%%%
\section{REINFORCEMENT LEARNING WITH SAC}

\subsection{Motivation}

Reinforcement learning offers the promise of discovering optimal strategies through interaction, without requiring expert knowledge. Unlike supervised learning, RL agents can potentially discover novel strategies that surpass human-designed heuristics. I chose Soft Actor-Critic (SAC) \cite{haarnoja2018soft} for its sample efficiency and stability in continuous action spaces, which seemed well-suited to billiards' 5-dimensional continuous action space.

The fundamental appeal of RL for billiards lies in its ability to learn from experience: an agent that plays millions of games should, in theory, develop intuition for shot selection, position play, and risk management that would be difficult to hand-code.

\subsection{SAC Architecture}

\subsubsection{State Representation}

I designed a 53-dimensional state encoding with semantic grouping to capture all game-relevant information while maintaining a fixed input size regardless of how many balls remain on the table.

The state vector consists of five components. First, the \textbf{cue ball} is represented by 3 dimensions encoding its position $(x, y)$ normalized to $[0, 1]$ and a binary pocketed flag. Second, \textbf{own balls} occupy 21 dimensions representing 7 balls with 3 features each (position and pocketed status), sorted by distance to the cue ball to provide implicit priority ordering. Third, \textbf{opponent balls} follow the same 21-dimensional structure. Fourth, the \textbf{8-ball} receives special 3-dimensional treatment due to its game-ending significance. Finally, \textbf{global information} uses 5 dimensions to encode remaining ball counts for both sides, turn indicator, and game phase (whether targeting the 8-ball).

The distance-based sorting provides implicit priority hints while maintaining permutation consistency across episodes. Pocketed balls are encoded with position $(0, 0)$ and pocketed flag $1$ to distinguish them from balls near the origin.

\subsubsection{Network Architecture}

The \textbf{Actor} implements a Gaussian policy that outputs a probability distribution over actions:
\begin{equation}
\pi_\theta(a|s) = \mathcal{N}(\mu_\theta(s), \sigma_\theta(s)^2)
\end{equation}
The network consists of three hidden layers with dimensions [256, 256, 128] and ReLU activations, outputting mean and log-variance for each of the 5 action dimensions: velocity $V_0$, horizontal angle $\phi$, vertical angle $\theta$, and spin parameters $a$ and $b$.

The \textbf{Critic} employs twin Q-networks to mitigate overestimation bias, a common issue in value-based RL. The target Q-value is computed as:
\begin{equation}
Q_{target} = r + \gamma \min_{i=1,2} Q_{\phi_i'}(s', a') - \alpha \log \pi(a'|s')
\end{equation}
where $\gamma = 0.99$ is the discount factor and $\alpha$ is the automatically-tuned temperature parameter that balances exploration and exploitation.

\subsection{Reward Engineering}

Reward design proved to be one of the most challenging aspects of this project. I developed a multi-component reward function to provide denser feedback than simple win/loss signals:
\begin{equation}
r = r_{pocket} + r_{turn} + r_{foul} + r_{defense} + r_{terminal}
\end{equation}

The \textbf{pocketing reward} uses a proportional value system to evaluate ball-pocketing outcomes at different game stages. A critical design requirement is that a single shot may pocket both own balls and opponent balls simultaneously, and the reward function must properly evaluate such mixed outcomes.

The reward for pocketing own balls uses proportional scaling:
\[
r_{own} = \frac{n_{pocketed}}{n_{remaining}} \times C_1
\]
where $C_1 = 100$ is a scaling constant. Similarly, the penalty for pocketing opponent balls uses proportional scaling:
\[
r_{opponent} = -\frac{n_{opponent\_pocketed}}{n_{opponent\_remaining}} \times C_2
\]
where $C_2 = 100$. The total pocketing reward is the sum:
\[
r_{pocket} = r_{own} + r_{opponent}
\]

This proportional design serves two purposes. First, it reflects stage-dependent strategic value: pocketing a ball when only 2 remain (50\% progress) provides twice the reward as pocketing when 4 remain (25\% progress), reflecting the increased importance of each ball as the game progresses. Second, it enables fair evaluation of mixed outcomes: if a shot pockets 1 own ball (from 4 remaining) and 1 opponent ball (from 3 remaining), the net reward is
\[
\frac{1}{4} \times 100 - \frac{1}{3} \times 100 = 25 - 33.3 = -8.3
\]
correctly signaling that helping the opponent clear their balls outweighs the benefit of pocketing one's own ball in this scenario.

The \textbf{turn dynamics} component provides $+10$ for retaining turn (successful pocket or opponent foul) and $-5$ for losing turn, encouraging aggressive play that maintains initiative.

The \textbf{foul penalties} are graded by severity: scratch (pocketing cue ball) receives $-20$, wrong ball first contact receives $-15$, no rail contact after cue ball hits object ball receives $-10$, and complete miss (no contact) receives $-25$.

The \textbf{defense reward} was designed to provide $+5$ retroactively when the opponent fails on their turn (misses or fouls), encouraging defensive positioning. However, this component was not used in the actual training experiments. The rationale was that defense rewards are only meaningful against strong opponents who can capitalize on positioning mistakes; against weaker opponents in self-play or simplified scenarios, defensive positioning provides minimal advantage. Since the RL agent never progressed to training against strong opponents, this reward component remained disabled in all experiments.

The \textbf{terminal rewards} for game outcomes were a critical design variable. I experimented with both large ($\pm 1000$) and compact ($\pm 100$) scales, as well as different draw penalties, which proved to be the most consequential design choice.

\subsection{Initial Approach: Training Against BasicAgent}

My initial strategy followed a curriculum learning approach with three stages of increasing difficulty. Stage 1 (15,000 episodes) focused on training against BasicAgent only, targeting greater than 70\% win rate. The goal was to learn legal shots and basic pocketing before facing stronger opponents. Stage 2 (25,000 episodes) introduced mixed opponents: 60\% BasicAgent, 30\% PhysicsAgent, and 10\% self-play, targeting greater than 40\% win rate against PhysicsAgent. Stage 3 (30,000 episodes) added MCTS opponents with the mixture 20\% BasicAgent, 30\% PhysicsAgent, 20\% MCTS, and 30\% self-play.

This curriculum was designed to gradually expose the agent to stronger opponents while building fundamental skills. However, the results were disappointing.

\subsubsection{Complete Failure Against BasicAgent}

When I removed the environment noise (which adds small perturbations to shot execution), the trained agent achieved \textbf{0\% win rate} against BasicAgent. The agent could not pocket a single ball reliably.

When environment noise was enabled, the agent occasionally won games, but analysis revealed that \textbf{100\% of these wins came from opponent errors}---specifically, cases where BasicAgent accidentally pocketed the 8-ball prematurely or scratched. The agent itself contributed nothing to these victories.

This result led to a critical insight: BasicAgent, despite being a ``basic'' opponent, is far too strong as an initial teacher for RL. The agent needs to experience successful pocketing to learn, but against BasicAgent, random exploration almost never produces successful outcomes. Without positive reward signals, the agent cannot learn which actions are good.

\subsection{Pivot to Self-Play with Ball Count Curriculum}

Given that BasicAgent was too strong, I redesigned the training approach around pure self-play with a curriculum based on the number of balls on the table.

\subsubsection{Ball Count Curriculum Design}

Instead of varying opponent difficulty, I varied game complexity by controlling the number of balls. The \textbf{4-Ball Stage} represents the simplest possible game: one cue ball, one own ball, one opponent ball, and the 8-ball. This configuration provides the maximum probability of successful pocketing through random exploration, while still requiring the agent to learn the fundamental rule that the 8-ball must be pocketed last. The \textbf{6-Ball Stage} introduces two own balls and two opponent balls, plus cue and 8-ball, adding more complex decision-making about shot selection. The eventual goal was to train on the complete 16-ball game.

For consistency in state representation, balls not present in simplified scenarios are marked as ``already pocketed'' in the state encoding, maintaining the 53-dimensional input throughout all stages.

\subsubsection{Discovery of the Degenerate Strategy}

Training on the 4-ball scenario with self-play revealed an unexpected and frustrating phenomenon. Rather than learning to pocket balls aggressively, the agent converged to a degenerate strategy.

The game has a 60-shot limit, after which the winner is determined by who has pocketed more balls. The agent discovered that attempting to pocket balls carries risk: a missed shot might leave an easy opportunity for the opponent, and prematurely pocketing the 8-ball results in immediate loss. The ``optimal'' strategy in symmetric self-play became: avoid pocketing any balls, run out the clock, and hope for a favorable tie-breaker.

\begin{table}[h]
\caption{Self-Play with Large Rewards ($\pm$1000): Convergence to Degenerate Strategy}
\label{tab:exp1}
\begin{center}
\begin{tabular}{|r|c|c|c|c|}
\hline
\textbf{Games} & \textbf{A Win} & \textbf{B Win} & \textbf{Draw} & \textbf{Avg Pockets} \\
\hline
800 & 45\% & 46\% & 9\% & 0.53 \\
8,000 & 26\% & 26\% & 48\% & 0.25 \\
200,000 & 25\% & 25\% & \textbf{51\%} & 0.32 \\
\hline
\end{tabular}
\end{center}
\end{table}

Table~\ref{tab:exp1} shows this phenomenon quantitatively. Early in training (800 games), draws occurred in only 9\% of games, with active pocketing. But as training progressed, the draw rate climbed to 51\% by 200,000 games, with average pockets per game dropping from 0.53 to 0.32.

\subsubsection{Reward Modifications to Counter Degenerate Strategy}

I experimented with several reward modifications to discourage the degenerate strategy.

\textbf{Compact Rewards ($\pm$100)} tested whether smaller reward magnitudes would reduce the perceived risk of aggressive play. As shown in Table~\ref{tab:exp2}, the results were nearly identical---47\% draw rate after 20,000 games.

\begin{table}[h]
\caption{Compact Rewards ($\pm$100): Same Degenerate Convergence}
\label{tab:exp2}
\begin{center}
\begin{tabular}{|r|c|c|c|c|}
\hline
\textbf{Games} & \textbf{A Win} & \textbf{B Win} & \textbf{Draw} & \textbf{Active Rate} \\
\hline
800 & 46\% & 44\% & 10\% & 90\% \\
8,000 & 27\% & 27\% & 46\% & 54\% \\
20,000 & 26\% & 26\% & \textbf{47\%} & 52\% \\
\hline
\end{tabular}
\end{center}
\end{table}

\textbf{Strict Draw Penalty} modified the reward function to penalize draws as severely as losses (draw penalty = $-100$, same as loss). This was the most successful intervention, reducing the draw rate to 36\% and increasing the active play rate to 64\%.

\begin{table}[h]
\caption{Strict Draw Penalty: Reduced Degenerate Behavior}
\label{tab:exp3}
\begin{center}
\begin{tabular}{|r|c|c|c|c|}
\hline
\textbf{Games} & \textbf{A Win} & \textbf{B Win} & \textbf{Draw} & \textbf{Active Rate} \\
\hline
800 & 45\% & 44\% & 11\% & 89\% \\
8,000 & 25\% & 26\% & 49\% & 51\% \\
20,000 & 32\% & 32\% & \textbf{36\%} & \textbf{64\%} \\
\hline
\end{tabular}
\end{center}
\end{table}

With the strict draw penalty, the agent finally learned to play the 4-ball game with reasonable competence, achieving approximately 50-50 win rates in self-play and averaging about 25 shots per game. This was encouraging---the agent had learned something about pocketing balls.

\begin{figure}[h]
      \centering
\includegraphics[width=0.95\columnwidth]{figures/sac_experiments_comparison.pdf}
\caption{SAC training curves comparing three reward configurations. The strict draw penalty successfully reduced the draw rate and encouraged more aggressive play, though all configurations show convergence toward conservative equilibria.}
\label{fig:sac_curves}
   \end{figure}
   
\subsubsection{Failure to Scale: The 6-Ball Collapse}

Encouraged by success in the 4-ball scenario, I attempted to train on the 6-ball configuration. The results were disheartening: the agent immediately reverted to the degenerate strategy, despite the strict draw penalty that had worked for 4 balls.

I experimented with numerous reward function modifications: increasing pocketing rewards, adding intermediate progress bonuses, penalizing low-velocity shots, and rewarding ball proximity to pockets. None of these interventions prevented convergence to conservative play.

\subsection{Root Cause Analysis}

Through this extensive experimentation, I identified several fundamental reasons why reinforcement learning struggles with billiards.

\subsubsection{The Sequential Constraint Problem}

Billiards has a critical sequential constraint: the 8-ball must be pocketed \textbf{last}, after all own balls are cleared. Pocketing the 8-ball prematurely results in immediate loss. This creates a severe reward design challenge.

The agent must learn to pocket own balls aggressively while avoiding the 8-ball entirely until the endgame. But how can a reward function communicate this temporal constraint? If we penalize 8-ball contact, the agent becomes overly conservative. If we don't, the agent frequently loses by premature 8-ball pockets.

In the 4-ball scenario with only one own ball, this constraint is manageable. But with more balls, the agent must plan multi-shot sequences where early decisions affect late-game 8-ball positioning---a level of temporal abstraction that appears beyond current RL capabilities without explicit planning.

\subsubsection{Sparse Rewards and Credit Assignment}

Meaningful feedback occurs only after complete physics simulation of each shot. There is no gradient signal for ``almost good'' shots---a ball that stops one inch from a pocket provides the same reward as a complete miss. With games lasting 20+ shots, attributing terminal rewards to specific early-game decisions is extremely difficult.

\subsubsection{High Precision Requirements}

A $1^\circ$ error in the shooting angle $\phi$ can mean the difference between pocketing a ball and missing entirely. The reward landscape is essentially discontinuous: tiny changes in action produce dramatically different outcomes. This makes gradient-based optimization extremely challenging.

\subsubsection{Exploration Inefficiency}

Random Gaussian exploration in the 5-dimensional action space almost never produces successful pockets. The probability of randomly selecting an action that pockets a ball is vanishingly small, so the agent receives almost no positive reward signals during early training.

\subsection{Lessons Learned from RL}

The RL experiments, despite their ultimate failure, yielded valuable insights. First, \textbf{reward structure matters more than magnitude}: the strict draw penalty was more effective than scaling rewards by 10x. Second, \textbf{learned rules do not generalize to increased complexity}: while the agent learned to respect the 8-ball-last rule in the 4-ball scenario, this sequential constraint knowledge failed to transfer when scaling to 6 balls, causing immediate regression to degenerate play. Third, \textbf{sequential constraints are hard to learn}: the 8-ball-last rule requires temporal reasoning that pure RL struggles to acquire. Fourth, \textbf{self-play finds Nash equilibria, not optimal play}: symmetric training converges to stable but potentially suboptimal strategies.

The fundamental conclusion is sobering: for high-precision continuous control with binary outcomes and complex sequential constraints, end-to-end RL requires prohibitive sample complexity. When a fast, accurate simulator is available, search-based methods that can explicitly evaluate action consequences are far more practical.

%%%%%%%%%%%%%%%%%%%%%%%%%%%%%%%%%%%%%%%%%%%%%%%%%%%%%%%%%%%%%%%%%%%%%%%%%%%%%%%%
\section{IMITATION LEARNING}

Given the fundamental challenges encountered with reinforcement learning, I pivoted to imitation learning. However, the primary motivation was not merely the failure of RL, but a practical constraint: the competition requires real-time decision-making with a strict time limit of 3 minutes per game (1.5 minutes per player). While my teammate had developed an excellent MCTS agent with 400 simulations and 64 candidate actions per decision, achieving 96.9\% win rate against BasicAgent, this strong policy exceeded the time budget during actual gameplay. To address this limitation, I designed an imitation learning pipeline to directly learn from the MCTS expert, with the goal of approximating its decision quality while achieving real-time inference speed through a lightweight neural network.

The hypothesis was that by observing expert behavior, a neural network could learn to approximate the MCTS policy without requiring explicit reward signals, extensive exploration, or the computational overhead of tree search. This approach would enable fast, single-forward-pass decisions that respect the time constraints while maintaining competitive performance.

\subsection{Expert Data Collection}

\subsubsection{Data Collection Pipeline}

I implemented a parallel data collection system using Python's ProcessPoolExecutor to efficiently gather expert demonstrations. The expert agent was the long\_mcts implementation with 400 MCTS simulations and 64 candidate actions per decision, representing the strongest available policy. To generate diverse game states, the expert played against BasicAgent, which provided a reasonable opponent that creates varied board configurations.

The parallel infrastructure deployed 100 worker processes across a multi-core server, achieving a collection rate of approximately 100 games per hour. Each game generates multiple state-action pairs (one per expert decision), and over several weeks of continuous collection, the pipeline accumulated a dataset of \textbf{1,787,130 state-action pairs}. This represents one of the largest billiards imitation learning datasets reported in the literature.

\subsubsection{State Representation}

I designed an 80-dimensional state representation that captures all information relevant to shot selection. The \textbf{cue ball} occupies 3 dimensions encoding its $(x, y)$ position normalized to the table dimensions and a binary flag indicating whether it has been pocketed (which would indicate a foul state). The \textbf{15 object balls} each receive 3 dimensions for position and pocketed status, totaling 45 dimensions. Ball positions are normalized to $[0, 1]$ based on table dimensions, and pocketed balls are encoded as $(0, 0, 1)$ to distinguish them from balls that happen to be near the table origin.

The \textbf{target mask} provides 15 binary dimensions indicating which balls are valid targets for the current player. This is crucial information: in 8-ball pool, players can only legally target their assigned group (solids or stripes) plus the 8-ball when appropriate. The \textbf{pocket positions} contribute 12 dimensions encoding the $(x, y)$ coordinates of all 6 pockets, providing the network with explicit target locations rather than requiring it to learn pocket positions implicitly.

Finally, \textbf{game statistics} occupy 5 dimensions: counts of remaining own and opponent balls, a flag indicating whether the player is currently targeting the 8-ball, whether the 8-ball has been pocketed, and the total count of valid targets. These features provide high-level game state information that might influence shot selection strategy.

\subsubsection{Action Representation}

The action labels present a careful encoding challenge, particularly for angular quantities. The velocity $V_0 \in [0.5, 8.0]$ m/s is simply normalized to $[0, 1]$ through linear scaling. The horizontal shooting angle $\phi \in [0^\circ, 360^\circ)$ requires special treatment due to its circular nature: an angle of $359^\circ$ is close to $1^\circ$, but naive encoding would treat them as maximally distant. I addressed this by encoding $\phi$ as the tuple $(\sin\phi, \cos\phi)$, which maps the circular domain to a continuous 2D space where angular proximity is preserved as Euclidean distance.

The vertical angle $\theta \in [0^\circ, 90^\circ]$ (cue elevation) is normalized to $[0, 1]$, and the spin parameters $a, b \in [-0.5, 0.5]$ are kept in their original range. The complete action label is thus 6-dimensional: normalized velocity, sine and cosine of horizontal angle, normalized vertical angle, and two spin components.

\subsection{Network Architectures}

I experimented with five architectures of increasing sophistication, each attempting to address limitations discovered in the previous approach.

\subsubsection{Architecture 1: MLP Networks}

The initial approach used simple feedforward networks, varying capacity to understand the impact of model size. The \textbf{small} configuration used 4 hidden layers with 512 units each, totaling approximately 8 million parameters. The \textbf{large} configuration scaled to 6 hidden layers with 1024 units each, reaching 26 million parameters. Both used ReLU activations, batch normalization, and dropout for regularization.

The loss function required careful weighting across action components, as the horizontal angle $\phi$ is most critical for pocketing accuracy:
\begin{equation}
\mathcal{L} = \mathcal{L}_{V_0} + 2\mathcal{L}_{\phi} + \mathcal{L}_{\theta} + 0.5\mathcal{L}_{spin}
\end{equation}
The doubled weight on $\mathcal{L}_{\phi}$ reflects that angle errors have the most dramatic impact on shot outcomes. The angle loss includes a regularization term to encourage the predicted $(\sin\phi, \cos\phi)$ to lie on the unit circle:
\begin{equation}
\mathcal{L}_{\phi} = MSE(\sin\phi) + MSE(\cos\phi) + 0.1\|\sin^2\phi + \cos^2\phi - 1\|^2
\end{equation}

\subsubsection{Architecture 2: Transformer-based}

Motivated by the observation that billiards involves reasoning about relationships between balls, I experimented with a Transformer architecture that treats balls as a set of tokens. Each ball is embedded by concatenating its position encoding with a learned type embedding (cue ball, own ball, opponent ball, or 8-ball). These embeddings pass through multi-head self-attention layers with 8 attention heads, allowing the network to learn which ball relationships are relevant for shot selection.

The attended ball representations are aggregated through global average pooling, and the resulting vector feeds into separate prediction heads for each action component. The hypothesis was that attention would help the network focus on relevant balls rather than treating all 16 balls uniformly.

\subsubsection{Architecture 3: Discrete Angle Classification}

Analysis of early training runs revealed that the horizontal angle $\phi$ was the most challenging component to predict accurately. Even small angular errors (5--10 degrees) often mean the difference between pocketing a ball and missing entirely. Motivated by this observation, I converted $\phi$ prediction from regression to classification.

The 360-degree range is discretized into 72 bins of 5 degrees each. The network outputs a 72-dimensional softmax distribution, and training uses cross-entropy loss for $\phi$ while maintaining MSE loss for other action components. At inference time, the predicted angle is the center of the highest-probability bin. This approach avoids the ``averaging'' problem to some extent: the network can learn a multi-modal distribution over bins rather than being forced to output a single point estimate.

I added LayerNorm before the final classification layer and clamped logits to $[-10, 10]$ to prevent numerical instability during early training when the network's predictions are essentially random.

\subsubsection{Architecture 4: Data Augmentation}

Billiards exhibits geometric symmetries that can be exploited to augment training data. A mirrored version of any valid state-action pair is equally valid, effectively multiplying the dataset size. I implemented four transformations that together provide 4× data augmentation.

The \textbf{original} data is used unchanged. \textbf{X-mirror} reflects the table across the horizontal axis: all $y$ coordinates become $1-y$, and the shooting angle becomes $-\phi$. \textbf{Y-mirror} reflects across the vertical axis: $x$ becomes $1-x$, and $\phi$ becomes $180^\circ - \phi$. Finally, \textbf{180-degree rotation} applies both reflections: $(x, y)$ becomes $(1-x, 1-y)$, and $\phi$ becomes $\phi + 180^\circ$.

These transformations preserve the validity of state-action pairs while exposing the network to more diverse configurations during training.

\subsubsection{Architecture 5: Equivariant Network}

The most sophisticated architecture explicitly encodes the permutation symmetry of object balls. The ordering of balls in the input should not affect the output---swapping the positions of Ball 3 and Ball 7 in the state representation should not change the optimal action.

I designed a Set Attention architecture with four key components. The \textbf{BallEncoder} is a shared MLP that independently processes each ball's features, producing per-ball embeddings. The \textbf{SetAttentionBlocks} stack 8 layers of self-attention over the ball set, allowing information to flow between ball representations while maintaining permutation equivariance. \textbf{CrossAttention} allows the cue ball embedding to query all object ball embeddings, reflecting the asymmetric role of the cue ball in shot selection. Finally, \textbf{target-weighted pooling} aggregates ball representations using the target mask as attention weights, emphasizing balls that are valid targets for the current shot.

Separate prediction heads output $V_0$, discrete $\phi$ (72 bins), $\theta$, and spin parameters. This architecture explicitly encodes the inductive bias that ball identity should not matter, only ball positions and target status.

\subsection{Training Results}

Table~\ref{tab:il_results} summarizes the training outcomes across all architectures. The validation loss varies significantly across models, though direct comparison is complicated by the different loss functions used (MSE vs cross-entropy for discrete models).

\begin{table}[h]
\caption{Imitation Learning Training Results}
\label{tab:il_results}
\begin{center}
\begin{tabular}{|l|c|c|c|c|}
\hline
\textbf{Model} & \textbf{Params} & \textbf{Val Loss} & \textbf{Best Epoch} & \textbf{$\phi$ Acc} \\
\hline
Large MLP & 26M & 0.547 & 40 & - \\
Small MLP & 8M & 0.663 & 47 & - \\
Aug Small & 8M & 0.636 & 106 & - \\
Discrete 72 & 26M & 5.55 & 1 & - \\
Equivariant & 26M & 2.97 & 25 & 16\% \\
\hline
\end{tabular}
\end{center}
\end{table}

The Large MLP achieved the lowest validation loss (0.547) among continuous models, reaching its best performance at epoch 40. The smaller 8M parameter model showed higher loss (0.663), suggesting that model capacity matters for this task. Data augmentation provided modest improvement (0.636 vs 0.663 for the small model) but required significantly more training epochs (106 vs 47) to converge.

The discrete classification approach showed concerning behavior: the best validation loss occurred at epoch 1, with subsequent training only increasing the loss. This suggests that the classification formulation may not be well-suited to this problem, possibly because the discretization loses important information about action similarity.

The equivariant network achieved 16\% accuracy on the discrete $\phi$ prediction (where random guessing would yield $\sim$1.4\% for 72 bins). While this is substantially above chance, it translates to an expected angular error of approximately 15 bins or 75 degrees---far too large for reliable pocketing. The best model was evaluated against BasicAgent, achieving a \textbf{35.5\% win rate}, substantially below the expert's performance.

\subsection{The Multi-Modal Action Problem}

Through careful analysis of the training data and model predictions, the fundamental challenge became clear: \textbf{for any given state, multiple distinct actions can be equally optimal, and this multi-modality defeats standard supervised learning approaches}.

Consider a concrete example: a game state where either Ball 3 or Ball 7 could be pocketed with similar difficulty. The MCTS expert, being stochastic due to random tie-breaking in its search, might choose Ball 3 in some games and Ball 7 in others. These two choices correspond to completely different shooting angles---perhaps $\phi = 45^\circ$ for Ball 3 and $\phi = 220^\circ$ for Ball 7.

When training with MSE loss, the network is incentivized to minimize squared error across all examples. Given similar states with conflicting labels, the optimal prediction under MSE is the \textit{mean}:
\begin{equation}
\hat{\phi} = \frac{\phi_{Ball3} + \phi_{Ball7}}{2} = \frac{45^\circ + 220^\circ}{2} = 132.5^\circ
\end{equation}

This averaged angle of $132.5^\circ$ pockets \textit{neither} ball. The network achieves low training loss by predicting the geometric center of a multi-modal distribution, but this center is not a valid action. This is a fundamental limitation of behavior cloning with MSE loss when expert behavior exhibits inherent variability.

I attempted several approaches to mitigate this problem. Converting $\phi$ to discrete classification with 72 bins allows the network to output a distribution over angles rather than a point estimate. In principle, this could represent multi-modal predictions. However, the 16\% accuracy suggests that the network still struggles to identify the correct mode. Mixture Density Networks (MDNs) could explicitly model multiple modes by predicting parameters of a Gaussian mixture, but this adds significant complexity and requires careful tuning of the number of mixture components. The most principled solution would be target conditioning: if the network also received information about which ball to target, the action distribution would become unimodal. However, this would require the expert to output target information, which MCTS does not naturally provide.

\subsection{Overfitting and Distribution Shift}

All models showed classic overfitting patterns during training. The training loss continued decreasing monotonically, while validation loss plateaued early and eventually began increasing. The gap between training and validation performance widened steadily over epochs.

This behavior suggests that despite 1.78 million samples, the dataset does not adequately cover the state space for robust generalization. The billiards state space is effectively infinite---continuous ball positions in a high-dimensional space---and expert behavior varies based on subtle factors that may not be fully captured in our 80-dimensional encoding.

More insidiously, even if the network achieves low validation loss, it may fail catastrophically during actual gameplay due to distribution shift. The validation set consists of states generated by expert-vs-opponent gameplay, but when the trained network plays, it may visit states that never appear in training data. A single bad prediction can leave the game in an unusual configuration, leading to further unfamiliar states and compounding errors.

\subsection{Lessons Learned from Imitation Learning}

The imitation learning experiments, while unsuccessful in producing a strong agent, yielded several important insights for future work.

First, \textbf{multi-modal action distributions fundamentally defeat MSE-based behavior cloning}. When experts exhibit varied behavior for similar states, the learned policy predicts invalid averaged actions. This is not merely a training issue but a fundamental mismatch between the loss function and the problem structure.

Second, \textbf{angular precision requirements are extreme for billiards}. With 72 bins spanning 360 degrees, each bin covers 5 degrees. The observed 16\% classification accuracy implies expected errors of approximately 75 degrees, while successful pocketing typically requires sub-degree precision. Achieving sufficient angular accuracy through supervised learning would require either much finer discretization (with correspondingly more data) or a fundamentally different approach.

Third, \textbf{distribution shift causes compounding errors during deployment}. Even with 1.78 million training samples, the learned policy visits states outside the training distribution during gameplay, leading to increasingly poor predictions over the course of a game.

Fourth, \textbf{inductive biases help but do not solve the core problem}. The equivariant network architecture correctly encodes permutation symmetry, and the attention mechanism learns to focus on relevant balls. However, these architectural improvements cannot address the fundamental multi-modality issue.

%%%%%%%%%%%%%%%%%%%%%%%%%%%%%%%%%%%%%%%%%%%%%%%%%%%%%%%%%%%%%%%%%%%%%%%%%%%%%%%%
\section{COMPARATIVE ANALYSIS}

\begin{table}[h]
\caption{Agent Performance Comparison}
\label{tab:comparison}
\begin{center}
\begin{tabular}{|l|c|c|}
\hline
\textbf{Agent} & \textbf{vs Basic} & \textbf{vs Pro} \\
\hline
Physics-based & 60\% & $\sim$40\%* \\
SAC (RL) & Failed & - \\
Imitation Learning & 35.5\% & - \\
\hline
BasicAgentPro & 85.4\% & - \\
future (MCTS) & 57.3\% & 33.1\% \\
\textbf{long\_mcts} & \textbf{96.9\%} & \textbf{80.6\%} \\
\hline
\end{tabular}
\end{center}
\footnotesize{*Estimated}
\end{table}

The results demonstrate a clear hierarchy among approaches, with MCTS-based agents significantly outperforming all learning methods.

The imitation learning agent achieved only 35.5\% win rate against BasicAgent, substantially lower than the 96.9\% achieved by the MCTS expert it was trained to imitate. This dramatic performance gap directly illustrates the multi-modal action problem: the learned policy predicts ``averaged'' actions that correspond to neither of the expert's valid shot choices, resulting in frequent misses.

MCTS succeeds where learning methods fail for several fundamental reasons. First, MCTS provides \textbf{explicit look-ahead evaluation}: by simulating action consequences before execution, MCTS avoids the 18--25\% ``self-defeat'' rate observed in greedy baselines (Section II-B). The agent can reject shots that risk pocketing the 8-ball prematurely or scratching, something that RL struggled to learn implicitly.

Second, MCTS requires \textbf{no training phase}: it works directly from the physics engine without collecting millions of samples or tuning hyperparameters. This is a significant practical advantage when a fast, accurate simulator is available.

Third, MCTS \textbf{naturally handles multi-modality}: rather than learning a single averaged action, MCTS explicitly explores multiple shot options and selects the best one. This completely bypasses the averaging problem that plagues imitation learning with MSE loss.

Fourth, MCTS provides \textbf{robustness through noise testing}: by evaluating each candidate action with random perturbations, MCTS can identify shots that are robust to execution noise, leading to more reliable performance in practice.

%%%%%%%%%%%%%%%%%%%%%%%%%%%%%%%%%%%%%%%%%%%%%%%%%%%%%%%%%%%%%%%%%%%%%%%%%%%%%%%%
\section{CONCLUSION}

This project systematically explored learning-based approaches for billiards AI, ultimately demonstrating why search-based methods like MCTS remain superior for this domain.

\subsection{Summary of Approaches}

The \textbf{reinforcement learning} investigation developed a complete Soft Actor-Critic framework with careful reward engineering and curriculum learning. The initial approach of training against BasicAgent revealed that even this ``basic'' opponent is too strong for RL exploration---the agent achieved 0\% win rate when environment noise was removed, with all wins attributable to opponent errors rather than learned skill.

Pivoting to self-play with a ball count curriculum showed promise in the simplified 4-ball scenario. However, the agent discovered a reward-induced degenerate equilibrium: avoiding all pocketing attempts to run out the 60-shot clock. Strict draw penalties successfully countered this behavior for 4 balls, reducing draw rates from 51\% to 36\%. But scaling to 6 balls immediately caused regression to degenerate play, despite extensive reward function modifications.

The root cause analysis identified the sequential constraint problem as fundamental: the 8-ball-last rule requires temporal reasoning that pure RL struggles to learn implicitly. Combined with sparse rewards, high precision requirements, and exploration inefficiency, these challenges make end-to-end RL impractical for billiards without prohibitive computational resources.

The \textbf{imitation learning} investigation was motivated by the competition's real-time constraint: 3 minutes per game (1.5 minutes per player). While the MCTS expert achieved 96.9\% win rate, it exceeded the time budget. I collected 1.78 million state-action pairs from this expert and experimented with five increasingly sophisticated neural network architectures, aiming to achieve real-time inference while maintaining competitive performance. The Large MLP achieved the lowest validation loss, while the Equivariant Network with discrete angle classification achieved 16\% accuracy on the critical $\phi$ angle---substantially above random chance but far below the precision required for reliable pocketing.

The fundamental limitation proved to be multi-modal action distributions: when multiple balls can be targeted from similar states, MSE loss causes the network to predict averaged actions that pocket neither target. The best imitation learning agent achieved only 35.5\% win rate against BasicAgent, compared to the expert's 96.9\%.

\subsection{Key Insights}

The most important insight from this project is that \textbf{algorithm choice must match problem structure}. Billiards presents a combination of challenges---high precision requirements, sparse rewards, sequential constraints, and multi-modal optimal actions---that current learning methods struggle to address simultaneously.

When a fast, accurate physics simulator is available, search-based methods like MCTS enjoy decisive advantages. They can explicitly evaluate action consequences before commitment, naturally explore multiple shot options without averaging, and require no training phase or sample collection. The simulator provides the ``gradient signal'' that RL lacks: direct feedback on whether a proposed action would succeed.

The 18--25\% self-defeat rate observed in greedy baselines (Section II-B) further illustrates the value of look-ahead: even competent shot-making is insufficient without considering game-ending foul risks. MCTS addresses this naturally through simulation, while RL must somehow learn to avoid these rare but catastrophic events from sparse terminal rewards.

\subsection{Limitations and Future Work}

Several limitations of this work suggest directions for future research. The RL experiments used a simplified 2D action space (velocity and horizontal angle only); the full 5D action space including elevation and spin might enable qualitatively different strategies. The imitation learning pipeline could be extended with target conditioning---providing the network with explicit information about which ball to target---which would convert the multi-modal problem into a unimodal one.

More fundamentally, hybrid approaches that combine learning with search might achieve the best of both worlds. A learned value function could guide MCTS search toward promising regions of the action space, potentially reducing the computational cost of pure search while maintaining its robustness advantages. Alternatively, a learned policy could serve as a fast approximation for deployment, with MCTS used only for difficult decisions.

\subsection{Concluding Remarks}

The negative results in this project are themselves valuable contributions. They empirically demonstrate the boundaries of current learning methods on a concrete, well-defined task, and highlight the importance of careful algorithm selection based on problem characteristics. For high-precision continuous control with sparse rewards and sequential constraints, search-based methods remain the practical choice when simulation is available.

%%%%%%%%%%%%%%%%%%%%%%%%%%%%%%%%%%%%%%%%%%%%%%%%%%%%%%%%%%%%%%%%%%%%%%%%%%%%%%%%
\section*{CONTRIBUTIONS}

\subsection*{My Contributions}

I was responsible for the learning-based approaches and infrastructure development in this project, contributing the following components:

\textbf{Physics-based agent baseline}: I implemented a geometric shot calculation agent that computes ghost ball positions, evaluates cut angles, and uses physics simulation to verify shot feasibility. This agent achieved 60\% win rate against BasicAgent and provided the foundation for understanding the problem requirements.

\textbf{Opponent robustness analysis}: I designed and executed experiments using a RandomAgent to probe the robustness of baseline opponents. This analysis revealed that BasicAgent loses 25\% and BasicAgentPro loses 18\% of games through self-defeat (premature 8-ball pocketing, scratches), even against a nearly-passive opponent. This insight motivated the exploration of look-ahead methods.

\textbf{Complete SAC reinforcement learning framework}: I developed a full Soft Actor-Critic implementation adapted for billiards, including the 53-dimensional state encoding, multi-component reward function, and curriculum learning infrastructure. The framework supports parallel training across multiple CPU workers, with careful handling of physics engine compatibility requirements.

\textbf{Controlled RL experiments}: I designed and executed three systematic experiments comparing reward configurations (large vs compact terminal rewards, progressive vs strict draw penalties). These experiments demonstrated that reward structure matters more than magnitude and identified the degenerate equilibrium phenomenon in self-play training.

\textbf{Imitation learning pipeline}: I built the complete data collection infrastructure using parallel ProcessPoolExecutor workers, accumulating 1.78 million expert state-action pairs. I designed the 80-dimensional state representation and 6-dimensional action encoding, and implemented five neural network architectures of increasing sophistication: MLP variants, Transformer-based attention, discrete angle classification, geometric data augmentation, and equivariant Set Attention networks.

\textbf{Parallel evaluation framework}: I developed a robust parallel game simulation system for evaluating agents at scale, including output suppression for clean logging, timeout handling for stuck games, and detailed statistics collection. This framework enabled the systematic comparison of all agent variants.

\textbf{Failure mode analysis}: I conducted comprehensive analysis of why learning methods fail for billiards, identifying the multi-modal action problem in imitation learning and the sequential constraint challenge in reinforcement learning. These insights contribute to understanding when learning-based approaches are appropriate for control problems.

\subsection*{Teammate's Contributions}

My teammate focused on the search-based approach, contributing:

\textbf{MCTS agent development}: Design and implementation of the Monte Carlo Tree Search agent with configurable search depth, candidate count, and noise robustness testing.

\textbf{Agent optimization}: Iterative improvements to MCTS parameters and evaluation heuristics, culminating in the long\_mcts configuration with 400 simulations and 64 candidates.

\textbf{Final solution}: The final MCTS agent (long\_opening variant incorporating a safe opening strategy) achieved 96.2\% win rate against BasicAgent and 79.5\% against BasicAgentPro, representing our team's final submission.

%%%%%%%%%%%%%%%%%%%%%%%%%%%%%%%%%%%%%%%%%%%%%%%%%%%%%%%%%%%%%%%%%%%%%%%%%%%%%%%%
\begin{thebibliography}{99}

\bibitem{haarnoja2018soft} T. Haarnoja, A. Zhou, P. Abbeel, and S. Levine, ``Soft Actor-Critic: Off-Policy Maximum Entropy Deep Reinforcement Learning with a Stochastic Actor,'' in \textit{Proc. ICML}, 2018.

\bibitem{ross2011reduction} S. Ross, G. Gordon, and D. Bagnell, ``A Reduction of Imitation Learning and Structured Prediction to No-Regret Online Learning,'' in \textit{Proc. AISTATS}, 2011.

\bibitem{silver2016mastering} D. Silver et al., ``Mastering the Game of Go with Deep Neural Networks and Tree Search,'' \textit{Nature}, vol. 529, pp. 484-489, 2016.

\bibitem{zaheer2017deep} M. Zaheer et al., ``Deep Sets,'' in \textit{Proc. NeurIPS}, 2017.

\end{thebibliography}

%%%%%%%%%%%%%%%%%%%%%%%%%%%%%%%%%%%%%%%%%%%%%%%%%%%%%%%%%%%%%%%%%%%%%%%%%%%%%%%%
\section*{APPENDIX}

\subsection*{A. Hyperparameters}
\label{app:hyperparams}

\begin{table}[h]
\caption{SAC Training Hyperparameters}
\label{tab:sac_hyperparams}
\begin{center}
\begin{tabular}{|l|c|}
\hline
\textbf{Parameter} & \textbf{Value} \\
\hline
\multicolumn{2}{|c|}{\textit{Network Architecture}} \\
\hline
Actor hidden layers & [256, 256, 128] \\
Critic hidden layers & [256, 256, 128] \\
Activation & ReLU \\
\hline
\multicolumn{2}{|c|}{\textit{SAC Algorithm}} \\
\hline
Learning rate (actor/critic/$\alpha$) & $3 \times 10^{-4}$ \\
Discount factor $\gamma$ & 0.99 \\
Soft update rate $\tau$ & 0.005 \\
Initial temperature $\alpha_0$ & 1.0 \\
Target entropy & $-\dim(\mathcal{A})$ \\
\hline
\multicolumn{2}{|c|}{\textit{Replay Buffer}} \\
\hline
Buffer size & 100,000--200,000 \\
Batch size & 512 \\
Min buffer before training & 2,000--5,000 \\
Priority exponent $\alpha$ & 0.6 \\
\hline
\multicolumn{2}{|c|}{\textit{Training}} \\
\hline
Parallel workers & 32 \\
Games per batch & 32 \\
Updates per batch & 4 \\
\hline
\end{tabular}
\end{center}
\end{table}

\begin{table}[h]
\caption{Reward Function Parameters}
\label{tab:reward_params}
\begin{center}
\begin{tabular}{|l|c|c|}
\hline
\textbf{Component} & \textbf{Compact} & \textbf{Large} \\
\hline
Win reward & +100 & +1000 \\
Loss reward & $-100$ & $-1000$ \\
Draw penalty (default) & progress-based & progress-based \\
Draw penalty (strict) & $-100$ & $-1000$ \\
\hline
Pocket own ball & $+\frac{n}{N} \times 100$ & $+\frac{n}{N} \times 100$ \\
Pocket opponent ball & $-50$ & $-50$ \\
\hline
Keep turn bonus & +10 & +10 \\
Lose turn penalty & $-5$ & $-5$ \\
\hline
Scratch foul & $-20$ & $-20$ \\
Wrong ball first & $-15$ & $-15$ \\
No rail contact & $-10$ & $-10$ \\
Complete miss & $-25$ & $-25$ \\
\hline
\end{tabular}
\end{center}
\end{table}

\begin{table}[h]
\caption{Imitation Learning Hyperparameters}
\label{tab:il_hyperparams}
\begin{center}
\begin{tabular}{|l|c|}
\hline
\textbf{Parameter} & \textbf{Value} \\
\hline
\multicolumn{2}{|c|}{\textit{Data Collection}} \\
\hline
Expert agent & long\_mcts \\
MCTS simulations & 400 \\
MCTS candidates & 64 \\
Total samples & 1,787,130 \\
Train/Val split & 80\%/20\% \\
\hline
\multicolumn{2}{|c|}{\textit{Large MLP}} \\
\hline
Hidden layers & [1024] $\times$ 6 \\
Parameters & 26M \\
\hline
\multicolumn{2}{|c|}{\textit{Equivariant Network}} \\
\hline
Ball embedding dim & 64 \\
Attention heads & 8 \\
Set attention layers & 8 \\
\hline
\multicolumn{2}{|c|}{\textit{Training}} \\
\hline
Optimizer & AdamW \\
Learning rate & $10^{-4}$ \\
Weight decay & $10^{-4}$ \\
Batch size & 512 \\
Epochs & 100--200 \\
LR scheduler & ReduceOnPlateau \\
\hline
\end{tabular}
\end{center}
\end{table}

\subsection*{B. State and Action Space Details}
\label{app:spaces}

\subsubsection*{SAC State Encoding (53 dimensions)}

\begin{table}[h]
\begin{center}
\begin{tabular}{|l|c|l|}
\hline
\textbf{Component} & \textbf{Dims} & \textbf{Description} \\
\hline
Cue ball & 3 & $(x, y, pocketed)$ \\
Own balls & 21 & 7 balls $\times$ 3, sorted by dist \\
Opponent balls & 21 & 7 balls $\times$ 3, sorted by dist \\
8-ball & 3 & $(x, y, pocketed)$ \\
Game info & 5 & counts, phase, turn \\
\hline
\textbf{Total} & \textbf{53} & \\
\hline
\end{tabular}
\end{center}
\end{table}

\subsubsection*{Action Space}

\textbf{Full action space} (5 dimensions):
\begin{itemize}
    \item $V_0 \in [0.5, 8.0]$ m/s: cue velocity
    \item $\phi \in [0^\circ, 360^\circ)$: horizontal angle
    \item $\theta \in [0^\circ, 90^\circ]$: vertical angle (elevation)
    \item $a \in [-0.5, 0.5]$: horizontal spin offset
    \item $b \in [-0.5, 0.5]$: vertical spin offset
\end{itemize}

\textbf{Simplified action space} (2 dimensions, used in experiments):
\begin{itemize}
    \item $V_0 \in [1.0, 6.0]$ m/s
    \item $\phi \in [0^\circ, 360^\circ)$
\end{itemize}
Fixed: $\theta = 0^\circ$, $a = b = 0$ (no spin).

\end{document}
